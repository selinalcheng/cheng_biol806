% Options for packages loaded elsewhere
\PassOptionsToPackage{unicode}{hyperref}
\PassOptionsToPackage{hyphens}{url}
\PassOptionsToPackage{dvipsnames,svgnames,x11names}{xcolor}
%
\documentclass[
  letterpaper,
  DIV=11,
  numbers=noendperiod]{scrartcl}

\usepackage{amsmath,amssymb}
\usepackage{iftex}
\ifPDFTeX
  \usepackage[T1]{fontenc}
  \usepackage[utf8]{inputenc}
  \usepackage{textcomp} % provide euro and other symbols
\else % if luatex or xetex
  \usepackage{unicode-math}
  \defaultfontfeatures{Scale=MatchLowercase}
  \defaultfontfeatures[\rmfamily]{Ligatures=TeX,Scale=1}
\fi
\usepackage{lmodern}
\ifPDFTeX\else  
    % xetex/luatex font selection
\fi
% Use upquote if available, for straight quotes in verbatim environments
\IfFileExists{upquote.sty}{\usepackage{upquote}}{}
\IfFileExists{microtype.sty}{% use microtype if available
  \usepackage[]{microtype}
  \UseMicrotypeSet[protrusion]{basicmath} % disable protrusion for tt fonts
}{}
\usepackage{xcolor}
\setlength{\emergencystretch}{3em} % prevent overfull lines
\setcounter{secnumdepth}{-\maxdimen} % remove section numbering
% Make \paragraph and \subparagraph free-standing
\ifx\paragraph\undefined\else
  \let\oldparagraph\paragraph
  \renewcommand{\paragraph}[1]{\oldparagraph{#1}\mbox{}}
\fi
\ifx\subparagraph\undefined\else
  \let\oldsubparagraph\subparagraph
  \renewcommand{\subparagraph}[1]{\oldsubparagraph{#1}\mbox{}}
\fi


\providecommand{\tightlist}{%
  \setlength{\itemsep}{0pt}\setlength{\parskip}{0pt}}\usepackage{longtable,booktabs,array}
\usepackage{calc} % for calculating minipage widths
% Correct order of tables after \paragraph or \subparagraph
\usepackage{etoolbox}
\makeatletter
\patchcmd\longtable{\par}{\if@noskipsec\mbox{}\fi\par}{}{}
\makeatother
% Allow footnotes in longtable head/foot
\IfFileExists{footnotehyper.sty}{\usepackage{footnotehyper}}{\usepackage{footnote}}
\makesavenoteenv{longtable}
\usepackage{graphicx}
\makeatletter
\def\maxwidth{\ifdim\Gin@nat@width>\linewidth\linewidth\else\Gin@nat@width\fi}
\def\maxheight{\ifdim\Gin@nat@height>\textheight\textheight\else\Gin@nat@height\fi}
\makeatother
% Scale images if necessary, so that they will not overflow the page
% margins by default, and it is still possible to overwrite the defaults
% using explicit options in \includegraphics[width, height, ...]{}
\setkeys{Gin}{width=\maxwidth,height=\maxheight,keepaspectratio}
% Set default figure placement to htbp
\makeatletter
\def\fps@figure{htbp}
\makeatother

\KOMAoption{captions}{tableheading}
\makeatletter
\@ifpackageloaded{caption}{}{\usepackage{caption}}
\AtBeginDocument{%
\ifdefined\contentsname
  \renewcommand*\contentsname{Table of contents}
\else
  \newcommand\contentsname{Table of contents}
\fi
\ifdefined\listfigurename
  \renewcommand*\listfigurename{List of Figures}
\else
  \newcommand\listfigurename{List of Figures}
\fi
\ifdefined\listtablename
  \renewcommand*\listtablename{List of Tables}
\else
  \newcommand\listtablename{List of Tables}
\fi
\ifdefined\figurename
  \renewcommand*\figurename{Figure}
\else
  \newcommand\figurename{Figure}
\fi
\ifdefined\tablename
  \renewcommand*\tablename{Table}
\else
  \newcommand\tablename{Table}
\fi
}
\@ifpackageloaded{float}{}{\usepackage{float}}
\floatstyle{ruled}
\@ifundefined{c@chapter}{\newfloat{codelisting}{h}{lop}}{\newfloat{codelisting}{h}{lop}[chapter]}
\floatname{codelisting}{Listing}
\newcommand*\listoflistings{\listof{codelisting}{List of Listings}}
\makeatother
\makeatletter
\makeatother
\makeatletter
\@ifpackageloaded{caption}{}{\usepackage{caption}}
\@ifpackageloaded{subcaption}{}{\usepackage{subcaption}}
\makeatother
\ifLuaTeX
  \usepackage{selnolig}  % disable illegal ligatures
\fi
\usepackage{bookmark}

\IfFileExists{xurl.sty}{\usepackage{xurl}}{} % add URL line breaks if available
\urlstyle{same} % disable monospaced font for URLs
\hypersetup{
  pdftitle={Wake me up when\ldots?: Understanding drivers of springtime awakening in eastern oysters (Crassostrea virginica)},
  pdfauthor={Selina Cheng},
  colorlinks=true,
  linkcolor={blue},
  filecolor={Maroon},
  citecolor={Blue},
  urlcolor={Blue},
  pdfcreator={LaTeX via pandoc}}

\title{Wake me up when\ldots?: Understanding drivers of springtime
awakening in eastern oysters (\emph{Crassostrea virginica})}
\author{Selina Cheng}
\date{}

\begin{document}
\maketitle

\subsection{Introduction}\label{introduction}

\emph{Objectives}

\subsection{Methods}\label{methods}

\emph{Data collection}

The bivalve monitoring system was carried out at Jackson Estuarine
Laboratory (JEL), Durham, NH, from November 2023 to September 2024. 15
bivalves consisting of oysters, mussels, and scallops were maintained in
a seawater flow-through table, located in a greenhouse attached to the
main laboratory. Water in the system came directly from the nearby Great
Bay Estuary. Though the water is coarsely filtered to keep out large
organisms, phytoplankton are permitted to pass through, providing a
natural food source for organisms in the system . The system was cleaned
weekly to prevent buildup of algae and other fouling organisms.

The bivalve gaping data collection system was composed of a custom-built
circuit board, run using an ELEGOO Mega 2560 microcontroller. The system
was connected to and powered by a laptop, where data were taken
continuously every 12 seconds using the serial terminal program
CoolTerm. 15 magnetic Hall effect sensors were attached to each bivalve
via cable to monitor each individual's shell gape behavior. The sensor
was glued to one valve of the shell at the lip (furthest point from the
hinge) using cyanoacrylate glue, while a small neodymium magnet was
glued to the other valve, directly opposite from the sensor. Hall effect
sensors read a voltage signal that is proportional to the strength of
the magnetic field surrounding it; as the bivalves open and close their
valves, the voltage detected by the sensor changes with the position of
the magnet on the other valve. The valve gaping data reflect the voltage
signal detected by the Hall effect sensors. Data were monitored weekly
and sensors and magnets were replaced and reattached as needed. Data
were subset to include only oysters with the most continuous data
streams that encompassed December 2023 through May 2024. Some data
streams were discontinued due to oyster mortality, while others were
unable to be analyzed due to sensor malfunction.

Temperature data were taken continuously in the system using an MX2202
Onset Pendant Temperature/Light Data Logger. Data were taken at a
ten-minute interval until May 12, 2024. Afterwards, data were taken at a
one-minute interval.

\emph{Data cleaning and manipulation}

\emph{Statistical analyses}

\subsection{Results}\label{results}

\subsection{Discussion}\label{discussion}



\end{document}
